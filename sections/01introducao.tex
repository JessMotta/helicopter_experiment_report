%SEÇÃO 1----------------------------------------------
\chapter{INTRODUÇÃO}
    Diferentes métodos constituem a prática da melhoria contínua. É de suma importância que os administradores conheçam estas ferramentas para que haja sempre redução de desperdícios, aumento da eficiência e controle dos processos \cite{entenda_doe}.

    Planejamento de Experimentos (\ac{DOE}) é uma das técnicas utilizadas para estudar um produto ou processo, e assim, identificar os fatores que mais influenciam seu comportamento. Através deste método deve-se obter a mais otimizada configuração para a construção da peça ou elaboração do procedimento \cite{oquee_doe}.

    O desenvolvimento de um experimento bem executado deve explicitar os fatores-chave do processo, assim como a combinação dos fatores que fazem o processo funcionar de maneira aceitável. A variabilidade do processo, ou seja, a diferença entre o que esperamos de algo e o que realmente acontece também é um ponto a ser observado pelo executor do \ac{DOE}.

    O resultado que determina uma característica ou elemento do experimento é chamado de \textbf{variável de resposta}. Por ser um método de abordagem repetitiva, é necessário realizar ciclos de testes para alcançar um bom resultado. Estes ciclos devem possuir três etapas: \textbf{Rastreamento} - fase de delimitação de variáveis e do campo de atuação; \textbf{Projeto fatorial completo} - fase de combinação de fatores e níveis de fatores e \textbf{Projeto de superfície} - Modelagem dos resultados obtidos.   

    O \ac{DOE} pode ser aplicado em duas situações: planejamento de experimentos e correção de processos defeituosos. Um processo desenvolvido desde o início com esta aplicação garante que sua produção e gestão sejam sempre melhoradas e tenham custos e tempo reduzidos. É uma ferramenta de melhoria contínua bastante eficaz, desde que se tome os devidos cuidados com as etapas do experimento \cite{oquee_doe}.


    
    
    


