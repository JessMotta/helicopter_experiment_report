\chapter{CONCLUSÃO}
\label{chap:conclusao}

Com base na pesquisa elaborada neste relatório, foi possível constatar que o experimento com helicóptero de papel possui eficiência na implementação dos conceitos relativos ao \ac{DOE}. Durante o processo experimental pôde-se observar a importância de uma escolha acertada para as variáveis de entrada e saída, bem como da aleatoriedade de cada teste ao evitar ao máximo, por exemplo, a interferência do vento durante a coleta de dados.

A partir da análise realizada é possível então inferir que o modelo de regressão linear encontrado na equação \ref{eq:tempo} é válido para o helicóptero de papel utilizado, com as variáveis \textit{altura}, \textit{ad\_direita} e \textit{ad\_esquerda} possuindo relevância para a resposta observada na variável de saída (\textit{tempo}), sendo sugerido a otimização destes fatores para a construção de um helicóptero que possua um tempo de voo maior.