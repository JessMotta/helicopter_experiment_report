\chapter{PLANEJAMENTO DE EXPERIMENTO COM VÁRIOS FATORES}
\label{chap:estudo}

Antes de realizar o experimento foi necessário um estudo prévio sobre o \ac{DOE}, neste planejamento determina-se quais as configurações são eficientes para determinado processo.
Segundo \cite{coleman1993systematic}  as etapas para o desenvolvimento de um Planejamento de Experimento na Indústria devem ser as seguintes:
\begin{itemize}
    \item Caracterização do problema.;
    \item Escolha dos fatores de influência e níveis, e listar restrições;
    \item Seleção das variáveis de resposta;
    \item Determinação de um modelo de planejamento de experimento;
    \item Condução do experimento;
    \item Análise dos dados;
    \item Conclusões e recomendações.
\end{itemize}
Nesta seção será avaliado cada item mencionado.

\subsection{Caracterização do problema}
Nesta etapa é necessário desenvolver as ideias acerca do problema e sobre os objetivos específicos do experimento. É  fundamental a participação de toda a equipe como qualidade, engenharia, clientes e operadores, fazer um relato preciso sobre o problema para que dessa forma possa ser possível a compreensão do processo e uma possível solução do problema.

\subsection{Escolha dos fatores de influência e níveis, e listar restrições}
Os fatores de influência e os níveis são escolhidos após se obter uma boa definição do problema e a elaboração do objetivo do experimento. O responsável pelo experimento deverá determinar quais fatores devem variar, os intervalos nos quais esses fatores variarão e os níveis em que cada rodada será realizada. Quando se tem por objetivo fazer uma varredura dos fatores ou caracterização do processo, melhor manter baixo o número de níveis, geralmente dois níveis. É fundamental a investigação de todos os fatores que possam ser importantes.

É necessário listar e rotular as interações conhecidas e supostas, e as restrições no experimento, como métodos de aquisição de dados, duração, materiais, facilidade de alterar a variável de controle, tipo de experimento, etc.


\subsection{Seleção das variáveis de resposta}
Nesta etapa o responsável pelo experimento irá escolher a variável que fornece informação útil sobre o processo. Geralmente, tem-se como variável de resposta a média ou o désvio padrão, ou ambos, da característica medida.
A capacidade do medidor também interferirá nessa etapa pois caso seja baixa, apenas efeitos grandes serão detectados, ou será necessária replicação do experimento. O embasamento para selecionar a variável reposta vem da teoria, especialistas, experiência e até experimentos anteriores.

\subsection{Determinação de um modelo de planejamento de experimento}
A escolha do planejamento leva em consideração o tamanho da amostra, seleção de uma ordem adequada de rodadas para as tentativas experimentais, ou se a formação de blocos, que é a formação de unidades experimentais homogêneos, ou outras restrições de aleatorização,  que é a alocação do material
experimental às diversas condições de experimentação, quanto a ardem segundo a qual os ensaios individuas do experimento serão realizados, são determinados ao acaso, estão envolvidas. Exemplos de tipos de planejamento são: Fatorial, Completamente aleatorizado com um único fator, Fatorial 2k em blocos, Fatorial 2k fracionário, Blocos aleatorizados, Blocos incompletos balanceados, blocos incompletos parcialmente balanceados, Quadrados latinos, Quadrados de Youden, Hierárquico e Superfície de resposta.


\subsection{Condução do experimento}
Nesta etapa é de extrema importância o monitoramento do processo de forma a garantir que seja feito de acordo com o que foi planejado. Os erros no procedimento experimental que ocorrem durante a condução do experimento destruirão a validade do mesmo. 

\subsection{Análise dos dados}
Para analisar os dados deve-se empregar os métodos estatísticos, para se obter resultados e conclusões, não opiniões. Se o experimento foi planejado e executado corretamente, logo, o tipo de método estatístico não será complicado. 

Geralmente utiliza-se as técnicas de apresentação e análise como ANOVA, regressão, plots e t test.


\subsection{Conclusões e recomendações}
Após os dados serem analisados, o experimento deverá apresentar conclusões práticas sobre os resultados e recomendar uma ação. Nessa etapa são usados métodos gráficos para passar os resultados para outras pessoas. Também devem ser realizados sequências de acompanhamento e testes de confirmação para validar as conclusões do experimento. 

