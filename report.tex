% --------------------------------------------------------------------------
% Report template for BIR projects
% Report template with support for Portuguese and English languages
% Change language {brazil or english} in \documentclass as per the examples
% This template has support for the ABNT citing format
% 
% Original version: jan/2019
% https://github.com/
% 
% Based on ABNTEX2 and the thesis template
% --------------------------------------------------------------------------
\documentclass[
%\DeclareUnicodeCharacter{200B}{}
% --------------------------------------------------------------------------
% classe memoir . options                                                   
12pt,					% tamanho da fonte
openright,				% cap. começam em pág ímpar (ins pág vazia caso preciso)
twoside,				% para impressão em verso e anverso. Oposto a oneside
a4paper,				% tamanho do papel
% --------------------------------------------------------------------------
% classe abntex2 . options                                                  
%chapter=TITLE,			% títulos de capítulos convertidos em letras maiúsc.
%section=TITLE,			% títulos de seções convertidos em letras maiúsc.
%subsection=TITLE,		% títulos de subseções convertidos em letras maiúsc.
%subsubsection=TITLE,	% títulos de subsubseções convertidos em letras maiúsc.
% --------------------------------------------------------------------------
% Opções de IDIOMA do pacote babel                                          
english,
brazil
]{ABNT/abntex2_report}
% --------------------------------------------------------------------------
% Pacotes básicos    
\usepackage{lmodern}			% Usa a fonte Latin Modern			
\usepackage[T1]{fontenc}		% Selecao de codigos de fonte.
\usepackage[utf8]{inputenc}		% Codificacao do documento (conversão automática dos acentos)
\usepackage{indentfirst}		% Indenta o primeiro parágrafo de cada seção.
\usepackage{color}				% Controle das cores
\usepackage{graphicx}			% Inclusão de gráficos
\usepackage{microtype} 			% para melhorias de justificação
\usepackage{lipsum}	
\usepackage[brazilian,hyperpageref]{backref} % páginas com citações na bibliog.
%\usepackage[alf,abnt-etal-list=0,abnt-etal-cite=3,abnt-emphasize=bf]{abntex2cite}
\usepackage[alf]{abntex2cite}
%	
\usepackage{lastpage}			% Usado pela Ficha catalográfica
%\usepackage{subfig}
\usepackage{supertabular}       % tabela na capa do documento
\usepackage{booktabs}
\usepackage[table,xcdraw]{xcolor}
\usepackage{adjustbox}
\usepackage{amssymb,amsmath,mathrsfs}
\usepackage{algorithm,algpseudocode}
\usepackage{pgfplots}
\usepackage{tikz}
\usepackage{titlesec}
\usepackage{ragged2e}
\usepackage{tocloft}
\usepackage{threeparttable}
\usepackage{etoolbox}
\usepackage[normalem]{ulem}
\usepackage{yaacro}
\usepackage[none]{verlab}
%\usepackage{fontspec}
%\setmainfont{Helvetica Light}
\usepackage{lscape}
%\usepackage[graphicx]{realboxes}
\usepackage{rotating}
\usepackage{wrapfig}
\usepackage{caption}
\usepackage{subcaption}
\usepackage{dirtytalk}
\usepackage{pdfpages}
\usepackage{threeparttable}
\usepackage{hyperref}
%\hypersetup{draft}
\usepackage{float}

%R - LATEX
\makeatletter
\def\maxwidth{ %
  \ifdim\Gin@nat@width>\linewidth
    \linewidth
  \else
    \Gin@nat@width
  \fi
}
\makeatother

\definecolor{fgcolor}{rgb}{0.345, 0.345, 0.345}
\newcommand{\hlnum}[1]{\textcolor[rgb]{0.686,0.059,0.569}{#1}}%
\newcommand{\hlstr}[1]{\textcolor[rgb]{0.192,0.494,0.8}{#1}}%
\newcommand{\hlcom}[1]{\textcolor[rgb]{0.678,0.584,0.686}{\textit{#1}}}%
\newcommand{\hlopt}[1]{\textcolor[rgb]{0,0,0}{#1}}%
\newcommand{\hlstd}[1]{\textcolor[rgb]{0.345,0.345,0.345}{#1}}%
\newcommand{\hlkwa}[1]{\textcolor[rgb]{0.161,0.373,0.58}{\textbf{#1}}}%
\newcommand{\hlkwb}[1]{\textcolor[rgb]{0.69,0.353,0.396}{#1}}%
\newcommand{\hlkwc}[1]{\textcolor[rgb]{0.333,0.667,0.333}{#1}}%
\newcommand{\hlkwd}[1]{\textcolor[rgb]{0.737,0.353,0.396}{\textbf{#1}}}%
\let\hlipl\hlkwb

\usepackage{framed}
\makeatletter
\newenvironment{kframe}{%
 \def\at@end@of@kframe{}%
 \ifinner\ifhmode%
  \def\at@end@of@kframe{\end{minipage}}%
  \begin{minipage}{\columnwidth}%
 \fi\fi%
 \def\FrameCommand##1{\hskip\@totalleftmargin \hskip-\fboxsep
 \colorbox{shadecolor}{##1}\hskip-\fboxsep
     % There is no \\@totalrightmargin, so:
     \hskip-\linewidth \hskip-\@totalleftmargin \hskip\columnwidth}%
 \MakeFramed {\advance\hsize-\width
   \@totalleftmargin\z@ \linewidth\hsize
   \@setminipage}}%
 {\par\unskip\endMakeFramed%
 \at@end@of@kframe}
\makeatother

\definecolor{shadecolor}{rgb}{.97, .97, .97}
\definecolor{messagecolor}{rgb}{0, 0, 0}
\definecolor{warningcolor}{rgb}{1, 0, 1}
\definecolor{errorcolor}{rgb}{1, 0, 0}
\newenvironment{knitrout}{}{} % an empty environment to be redefined in TeX
\usepackage{alltt}
\IfFileExists{upquote.sty}{\usepackage{upquote}}{}
\DeclareUnicodeCharacter{200B}{}
% --------------------------------------------------------------------------%
% Configurações do PDF final                                                
\definecolor{blue}{RGB}{41,5,195}
\makeatletter
\hypersetup{
	%pagebackref=true,
	pdftitle={\@title}, 
	pdfauthor={\@author},
	pdfsubject={\@title},
	%pdfsubject={\imprimirpreambulo},
	pdfcreator={LaTeX with abnTeX2},
	pdfkeywords={abnt}{latex}{abntex}{abntex2}{\imprimirpalavraschave}, 
	colorlinks=true,       		% false: boxed links; true: colored links
	linkcolor=blue,          	% color of internal links
	citecolor=blue,        		% color of links to bibliography
	filecolor=magenta,      	% color of file links
	urlcolor=blue,
	bookmarksdepth=4
}
%\makeatother
% --------------------------------------------------------------------------
% Posiciona figuras e tabelas no topo da página quando adicionadas sozinhas
% em um página em branco. Ver https://github.com/abntex/abntex2/issues/170
%\makeatletter
\setlength{\@fptop}{5pt} % Set distance from top of page to first float
\makeatother
% --------------------------------------------------------------------------
% Formatação                                                                
\newcommand\tab[1][1cm]{\hspace*{#1}}
\apptocmd{\thebibliography}{\justifying}{}{} 
\renewcommand{\ABNTEXsectionfont}{\bfseries}
\titlespacing*{\chapter}{0pt}{0pt}{12pt}
\titlespacing*{\section}{0pt}{6pt}{6pt}
\titlespacing*{\subsection}{0pt}{6pt}{6pt}
\titlespacing*{\subsubsection}{0pt}{6pt}{6pt}
% --------------------------------------------------------------------------
% Rearranja os finais de cada estrutura                                     
\algrenewtext{EndWhile}{\algorithmicend\ \algorithmicwhile}
\algrenewtext{EndFor}{\algorithmicend\ \algorithmicfor}
\algrenewtext{EndIf}{\algorithmicend\ \algorithmicif}
\algrenewtext{EndFunction}{\algorithmicend\ \algorithmicfunction}
% --------------------------------------------------------------------------
% Espaçamentos entre linhas e parágrafos                                    
\setlength{\parindent}{1.3cm} % linha
\setlength{\parskip}{0.2cm} % parágrafo, tente também \onelineskip
% --------------------------------------------------------------------------
% Informações de dados para CAPA e FOLHA DE ROSTO                           
\prodtecnica{001 / 2020}
\titulo{Planejamento de Experimentos (DOE) - TIMON-HM}
% \tiporelatorio{Parcial} 
% \nomeprojeto{Projeto}
\outrossubtitulos{~} % opcional
\autores{
	Jéssica Lima Motta\\
	Leonardo Mendes de Souza Lima\\
	Miguel Felipe Nery Vieira\\
	Vinícius José Gomes de Araujo Felismino\
}
% \newcommand{\autoresexternos}{
% 	John Marston\\
% 	Frank West\
% }
\local{Salvador\\Bahia, Brasil}
\data{Setembro de 2020}
% \classificacao{( ) Confidencial  (X) Restrito  ( )  Uso Interno  ( ) Público}
% \revisao{01}
% \tabelacutter{000} 
% \palavraschave{1. Manipulator. 2. Simulation. 3. Computer vision.}
% \classificacaoassunto{000} % Número de Classificação do assunto 
%\parceirologo{logos/x.png}
%------------------------------------------------------------------
% Finalização das configurações da capa
%
%
%------------------------------------------------------------------              
% Acrônimos :: Chamar no texto como \ac{DoF}                                
\begin{acgroupdef}[list=acronyms]
	% \acdef{DoF}{Degrees of Freedom}
	% \acdef{PoC}{Proof of Concept, em português Prova de Conceito}
	% \acdef{UUV}{Unmanned Underwater Vehicle, em português Veículo Subaquático Não-tripulado}
	% \acdef{AUV}{Autonomous Underwater Vehicle, em português Veículo Subaquático Autônomo}
	% \acdef{UVM}{Unmanned Vehicle Morphing}
	% \acdef{SLAM}{Simultaneous Localization and Mapping}
	% \acdef{ROV}{Remotely Operated Vehicle}
	% \acdef{SOTA}{Study Of The Art}
	% %
	%
	%
\end{acgroupdef}
% --------------------------------------------------------------------------
% Criação do sumário
\makeindex
%
\begin{document}
	\frenchspacing
	\imprimircapa
	% \imprimircatalografica
% --------------------------------------------------------------------------
% Resumo                                                         
\ABNTEXchapterfont\large\textbf{\resumoatitlename}
\begin{flushleft}
	\normalsize
	\justify
	\normalfont
		Este documento tem como objetivo aplicar os conceitos de planejamento de um experimento utilizando um modelo de helicóptero de papel. O propósito principal foi identificar quais são os fatores que influenciam seu tempo de queda e como estas variáveis podem melhorar o seu desempenho. Durante o processo, foi medido o seu tempo de queda em duas alturas diferentes, além disto, adesivos foram colados em seu corpo e um clipe foi adicionado em sua parte inferior a fim de verificar a influência da variação destes parâmetros no resultado final. Para realizar o estudo estatístico dos dados foi utilizada a ferramenta R, uma linguagem de programação voltada à manipulação, análise e visualização de dados.		
	
\end{flushleft}
\vspace*{1cm}
\newpage

% Tabela de conteúdo                                                        	
	\begin{flushleft}
		\ABNTEXchapterfont\Large\textbf{\MakeUppercase\glosariotitlename}
	\end{flushleft}
	%\pagebreak
	\vspace*{-36pt}
	\pdfbookmark[0]{\contentsname}{toc}
	\normalsize
	\normalfont
	\tableofcontents*
	\justify
% --------------------------------------------------------------------------
% Formatação, remover espaço depois dos títulos
	\setlength\beforechapskip{-24pt}
	\setlength\afterchapskip{12pt}
	\textual
	\pagestyle{plain}
	\normalsize
	\justify
	\normalfont
% --------------------------------------------------------------------------
% Conteúdo do relatório  
	%SEÇÃO 1----------------------------------------------
\chapter{INTRODUÇÃO}
    Planejamento de Experimentos ou DOE (\textit{Design of Experiments}), é a técnica usada para estudar um produto ou processo, e assim, identificar os fatores que mais influenciam seu comportamento. Através deste método deve-se obter a mais otimizada configuração para a construção da peça ou elaboração do procedimento.

    O desenvolvimento de um experimento bem executado deve explicitar os fatores-chave do processo, assim como a combinação dos fatores que fazem o processo funcionar de maneira aceitável. A variabilidade do processo, ou seja, a diferença entre o que esperamos de algo e o que realmente acontece também é um ponto a ser observado pelo executor do DOE.

    O DOE pode ser aplicado em duas situações: planejamento de experimentos e correção de processos defeituosos. Um processo desenvolvido desde o início com esta aplicação garante que sua produção e gestão sejam sempre melhoradas e tenham custos e tempo reduzidos. 



    
    
    



% Este documento tem como objetivo analisar um experimento estatístico sobre um modelo de helicóptero de papel. 
% Durante o processo, foi medido o seu tempo de queda em duas alturas diferentes, 1,30 m e 2,10 m, 
%  além disto, 
% para alterar o seu desempenho, pedaços de fita foram colados em seu corpo e hélices e um clipe foi adicionado em sua parte inferior a fim de verificar a influência da variação destes parâmetros no resultado final. Para variar o valor. 
% O procedimento resultou em trinta e duas combinações distintas conforme vistas na tabela \ref{tab:dados_experimento} .

% Para realizar o estudo estatístico dos dados foi utilizada a ferramenta R, uma linguagem de programação voltada à manipulação, 
% análise e visualização de dados.


    % \begin{table}[H]
    %     \centering
    %     \caption{Dados do experimento.}
    %     \begin{tabular}{|c|c|c|c|c|c|}
    %     \hline
    %     \rowcolor[HTML]{EFEFEF} 
    %     \textbf{Clipe} & \textbf{Altura} & \textbf{Ad\_top} & \textbf{Ad\_left} & \textbf{Ad\_right} & \textbf{Score} \\ \hline
    %     +              & -               & -                & -                 & -                  & 1,57           \\ \hline
    %     \rowcolor[HTML]{EFEFEF} 
    %     -              & -               & -                & -                 & -                  & 1,27           \\ \hline
    %     +              & +               & -                & -                 & -                  & 1,70           \\ \hline
    %     \rowcolor[HTML]{EFEFEF} 
    %     -              & +               & -                & -                 & -                  & 1,10           \\ \hline
    %     +              & +               & +                & -                 & -                  & 1,75           \\ \hline
    %     \rowcolor[HTML]{EFEFEF} 
    %     -              & +               & +                & -                 & -                  & 1,30           \\ \hline
    %     +              & -               & +                & -                 & -                  & 1,82           \\ \hline
    %     \rowcolor[HTML]{EFEFEF} 
    %     -              & -               & +                & -                 & -                  & 1,31           \\ \hline
    %     +              & +               & +                & -                 & +                  & 1,68           \\ \hline
    %     \rowcolor[HTML]{EFEFEF} 
    %     -              & +               & +                & -                 & +                  & 1,35           \\ \hline
    %     +              & -               & +                & -                 & +                  & 2,04           \\ \hline
    %     \rowcolor[HTML]{EFEFEF} 
    %     -              & -               & +                & -                 & +                  & 1,42           \\ \hline
    %     +              & -               & +                & +                 & +                  & 1,86           \\ \hline
    %     \rowcolor[HTML]{EFEFEF} 
    %     -              & -               & +                & +                 & +                  & 1,32           \\ \hline
    %     +              & +               & +                & +                 & +                  & 1,63           \\ \hline
    %     \rowcolor[HTML]{EFEFEF} 
    %     -              & +               & +                & +                 & +                  & 1,17           \\ \hline
    %     +              & -               & -                & +                 & +                  & 1,58           \\ \hline
    %     \rowcolor[HTML]{EFEFEF} 
    %     -              & -               & -                & +                 & +                  & 1,44           \\ \hline
    %     +              & +               & -                & +                 & +                  & 1,73           \\ \hline
    %     \rowcolor[HTML]{EFEFEF} 
    %     -              & +               & -                & +                 & +                  & 1,25           \\ \hline
    %     +              & +               & -                & -                 & +                  & 1,55           \\ \hline
    %     \rowcolor[HTML]{EFEFEF} 
    %     -              & +               & -                & -                 & +                  & 1,23           \\ \hline
    %     +              & -               & -                & -                 & +                  & 1,91           \\ \hline
    %     \rowcolor[HTML]{EFEFEF} 
    %     -              & -               & -                & -                 & +                  & 1,50           \\ \hline
    %     +              & -               & -                & +                 & -                  & 1,92           \\ \hline
    %     \rowcolor[HTML]{EFEFEF} 
    %     -              & -               & -                & +                 & -                  & 1,36           \\ \hline
    %     +              & +               & -                & +                 & -                  & 1,71           \\ \hline
    %     \rowcolor[HTML]{EFEFEF} 
    %     -              & +               & -                & +                 & -                  & 1,52           \\ \hline
    %     +              & +               & +                & +                 & -                  & 1,74           \\ \hline
    %     \rowcolor[HTML]{EFEFEF} 
    %     -              & +               & +                & +                 & -                  & 1,32           \\ \hline
    %     +              & -               & +                & +                 & -                  & 1,83           \\ \hline
    %     \rowcolor[HTML]{EFEFEF} 
    %     -              & -               & +                & +                 & -                  & 1,40           \\ \hline
    %     \end{tabular}
    %     \label{tab:dados_experimento}
    %     % \caption*{Autoria própria.}
    % \end{table}

	\chapter{PLANEJAMENTO DE EXPERIMENTO COM VÁRIOS FATORES}
\label{chap:estudo}

Antes de realizar o experimento foi necessário um estudo prévio sobre o \ac{DOE}. Neste planejamento determina-se quais as configurações são eficientes para determinado processo.
Segundo \cite{coleman1993systematic}  as etapas para o desenvolvimento de um Planejamento de Experimento na Indústria devem ser as seguintes:
\begin{itemize}
    \item Caracterização do problema;
    \item Escolha dos fatores de influência e níveis, e listar restrições;
    \item Seleção das variáveis de resposta;
    \item Determinação de um modelo de planejamento de experimento;
    \item Condução do experimento;
    \item Análise dos dados;
    \item Conclusões e recomendações.
\end{itemize}
Nesta seção será avaliado cada item mencionado.

\section{Caracterização do problema}
Nesta etapa é necessário desenvolver as ideias acerca do problema e sobre os objetivos específicos do experimento. É  fundamental a participação de toda a equipe de qualidade, engenharia, clientes e operadores, para fazer um relato preciso sobre o problema. Dessa forma será possível uma melhor compreensão do processo e a busca por uma provável uma solução para o problema.

\section{Escolha dos fatores de influência e níveis, e listar restrições}
Os fatores de influência e os níveis são escolhidos após se obter uma boa definição do problema e a elaboração do objetivo do experimento. O responsável pelo experimento deverá determinar quais fatores devem variar, os intervalos nos quais esses fatores variarão e os níveis em que cada rodada será realizada. Quando se tem por objetivo fazer uma varredura dos fatores ou caracterização do processo, e este ainda não está amadurecido, melhor manter baixo o número de níveis, geralmente dois \cite{melhorespraticas}. É fundamental a investigação de todos os fatores que possam ser importantes.

É necessário listar e rotular as interações conhecidas e supostas, e as restrições no experimento, como métodos de aquisição de dados, duração, materiais, facilidade de alterar a variável de controle, tipo de experimento, etc.


\section{Seleção das variáveis de resposta}
Nesta etapa o responsável pelo experimento irá escolher a(s) variável/variáveis que fornece informação útil sobre o processo. Geralmente, tem-se como variável de resposta a média ou o désvio padrão, ou ambos, da característica medida.
A capacidade do medidor também interferirá nessa etapa pois, caso seja baixa, apenas efeitos grandes serão detectados, ou será necessária replicação do experimento. O embasamento para selecionar a variável resposta pode vir da teoria, de especialistas ou da experiência.

\section{Determinação de um modelo de planejamento de experimento}
A escolha do planejamento leva em consideração o tamanho da amostra, seleção de uma ordem adequada de rodadas para as tentativas experimentais, e se a formação de blocos ou outras restrições de aleatorização estão envolvidas \cite{Experimentos}. Podem-se citar dentre os métodos de planejamento: Fatorial, Completamente aleatorizado com um único fator, Fatorial $2^{k}$ em blocos, Fatorial $2^{k}$ fracionário, Blocos aleatorizados, Blocos incompletos balanceados, blocos incompletos parcialmente balanceados, Quadrados latinos, Quadrados de Youden, Hierárquico e Superfície de resposta \cite{montgomery2013estatistica}.

\section{Condução do experimento}
Nesta etapa é de extrema importância o monitoramento do processo de forma a garantir que seja feito de acordo com o que foi planejado. Os erros no procedimento experimental que ocorrem durante a condução do experimento destruirão a validade do mesmo, requerendo uma repetição dos testes.

\section{Análise dos dados}
Para analisar os dados deve-se empregar métodos estatísticos como ANOVA, regressão, plots e t test,e com isso obter resultados e conclusões. Se o experimento foi planejado e executado corretamente, logo, não serão encontradas dificuldades nas análises decorrentes do tipo de método estatístico aplicado. 

\section{Conclusões e recomendações}
Após os dados serem analisados, o experimento deverá apresentar conclusões práticas sobre os resultados e recomendar uma ação. Nessa etapa são usados métodos gráficos para apresentar os resultados a outras pessoas envolvidas no processo. Devem ser realizadas sequências de acompanhamento, a análise dos resíduos decorrente do processo e testes de confirmação, para validar as conclusões do experimento. 


	\chapter{EXPERIMENTO}
\label{chap:experimento}

Para aplicar os conceitos vistos na seção \ref{chap:estudo}, foi proposto um desafio em que consiste em modelar um experimento contendo um helicóptero de papel. O objetivo deste desafio é observar como a saída desejada, neste caso o maior tempo de voo, está relacionada com as variáveis de entrada.  

Para conceber o protótipo do helicóptero para o estudo, foi utilizado o modelo proposto pela metodologia \textit{SixSigma}, conforme visto na Figura \ref{fig:model_heli}. 

\begin{figure}[H]
  \caption{Modelo do helicóptero de papel.}
  \centering
  \includegraphics[width=0.7\textwidth]{images/helicopter.jpeg}
  \label{fig:model_heli}
\end{figure}

Seguindo as recomendações do \textit{template}, foi obtido como modelo de configuração inicial, o helicóptero visto na Figura \ref{fig:heli_papel}. Este, apresenta as asas e o corpo com o comprimento máximo (9,5 cm), e que não deverá ser alterado. O seu tempo de voo é medido desde o momento em que é lançado da altura definida até o momento em que o mesmo atinge o solo.

% O tempo de voo é medido desde o momento em que o helicóptero é lançado da altura definida até o momento em que atinge o solo.    

% Após as dobras e cortes recomendados pelo \textit{template}, o protótipo obtido como configuração inicial para análise do estudo pode ser visto na Figura \ref{fig:heli_papel}.

\begin{figure}[H]
  \caption{Helicóptero de papel.}
  \centering
  \includegraphics[width=1\textwidth]{images/IMG_20200918_162251.jpg}
  \caption*{Fonte: Autoria própria.}
  \label{fig:heli_papel}
\end{figure}

Para realizar os testes, foram considerados alguns fatores que influenciam no tempo de voo, conforme vistos na Tabela \ref{tab:fatores}.

\begin{table}[H]
  \caption{Fatores considerados para alterar a estrutura.}
  \centering
  \begin{tabular}{|c|c|c|}
  \hline
  \rowcolor[HTML]{EFEFEF} 
  \textbf{Fatores}              & \textbf{Configuração atual} & \textbf{Alteração permitida} \\ \hline
  Comprimento (asa e corpo) (m) & 0,095                       & Não                          \\ \hline
  \rowcolor[HTML]{EFEFEF} 
  Clipe                         & Não                         & Sim                          \\ \hline
  \rowcolor[HTML]{FFFFFF} 
  Altura (m)                    & 1,30                        & 2,10                         \\ \hline
  \rowcolor[HTML]{EFEFEF} 
  Adesivo (Asa)                 & Não                         & Sim                          \\ \hline
  \rowcolor[HTML]{FFFFFF} 
  Adesivo (Corpo/Esquerdo)         & Não                         & Sim                          \\ \hline
  \rowcolor[HTML]{EFEFEF} 
  Adesivo (Corpo/Direito)          & Não                         & Sim                          \\ \hline
  \end{tabular}
  \caption*{Fonte: Autoria própria.}
  \label{tab:fatores}
  \end{table}

Por fim, foi construída a Tabela \ref{tab:dados_experimento} que contém o tempo de voo para cada uma das possíveis combinações dos fatores. Para as variáveis \textit{clipe}, \textit{Ad\_top}, \textit{Ad\_esquerda} e \textit{Ad\_direita} o simbolo ``\textbf{+}'' indica a sua presença enquanto o ``\textbf{-}'' representa a sua ausência, já para a variável \textit{Altura} o ``\textbf{+}'' retrata sua configuração inicial de 1,30 metros e o ``\textbf{-}'' representa a altura de 2,10 metros. O próximo passo é utilizar a ferramenta R para realizar o estudo de planejamento de experimentos (DOE) e analisar qual das configurações está exercendo uma maior influência no experimento, que será discutido na seção \ref{chap:resultados}.

\begin{table}[H]
  \centering
  \caption{Dados do experimento.}
  \begin{tabular}{|c|c|c|c|c|c|}
  \hline
  \rowcolor[HTML]{EFEFEF} 
  \textbf{Clipe} & \textbf{Altura} & \textbf{Ad\_top} & \textbf{Ad\_esquerda} & \textbf{Ad\_direita} & \textbf{Tempo} \\ \hline
  +              & -               & -                & -                 & -                  & 1,57           \\ \hline
  \rowcolor[HTML]{EFEFEF} 
  -              & -               & -                & -                 & -                  & 1,27           \\ \hline
  +              & +               & -                & -                 & -                  & 1,70           \\ \hline
  \rowcolor[HTML]{EFEFEF} 
  -              & +               & -                & -                 & -                  & 1,10           \\ \hline
  +              & +               & +                & -                 & -                  & 1,75           \\ \hline
  \rowcolor[HTML]{EFEFEF} 
  -              & +               & +                & -                 & -                  & 1,30           \\ \hline
  +              & -               & +                & -                 & -                  & 1,82           \\ \hline
  \rowcolor[HTML]{EFEFEF} 
  -              & -               & +                & -                 & -                  & 1,31           \\ \hline
  +              & +               & +                & -                 & +                  & 1,68           \\ \hline
  \rowcolor[HTML]{EFEFEF} 
  -              & +               & +                & -                 & +                  & 1,35           \\ \hline
  +              & -               & +                & -                 & +                  & 2,04           \\ \hline
  \rowcolor[HTML]{EFEFEF} 
  -              & -               & +                & -                 & +                  & 1,42           \\ \hline
  +              & -               & +                & +                 & +                  & 1,86           \\ \hline
  \rowcolor[HTML]{EFEFEF} 
  -              & -               & +                & +                 & +                  & 1,32           \\ \hline
  +              & +               & +                & +                 & +                  & 1,63           \\ \hline
  \rowcolor[HTML]{EFEFEF} 
  -              & +               & +                & +                 & +                  & 1,17           \\ \hline
  +              & -               & -                & +                 & +                  & 1,58           \\ \hline
  \rowcolor[HTML]{EFEFEF} 
  -              & -               & -                & +                 & +                  & 1,44           \\ \hline
  +              & +               & -                & +                 & +                  & 1,73           \\ \hline
  \rowcolor[HTML]{EFEFEF} 
  -              & +               & -                & +                 & +                  & 1,25           \\ \hline
  +              & +               & -                & -                 & +                  & 1,55           \\ \hline
  \rowcolor[HTML]{EFEFEF} 
  -              & +               & -                & -                 & +                  & 1,23           \\ \hline
  +              & -               & -                & -                 & +                  & 1,91           \\ \hline
  \rowcolor[HTML]{EFEFEF} 
  -              & -               & -                & -                 & +                  & 1,50           \\ \hline
  +              & -               & -                & +                 & -                  & 1,92           \\ \hline
  \rowcolor[HTML]{EFEFEF} 
  -              & -               & -                & +                 & -                  & 1,36           \\ \hline
  +              & +               & -                & +                 & -                  & 1,71           \\ \hline
  \rowcolor[HTML]{EFEFEF} 
  -              & +               & -                & +                 & -                  & 1,52           \\ \hline
  +              & +               & +                & +                 & -                  & 1,74           \\ \hline
  \rowcolor[HTML]{EFEFEF} 
  -              & +               & +                & +                 & -                  & 1,32           \\ \hline
  +              & -               & +                & +                 & -                  & 1,83           \\ \hline
  \rowcolor[HTML]{EFEFEF} 
  -              & -               & +                & +                 & -                  & 1,40           \\ \hline
  \end{tabular}
  \label{tab:dados_experimento}
  \caption*{Fonte: Autoria própria.}
\end{table}
	\chapter{INTERPRETAÇÃO DOS RESULTADOS OBTIDOS}
<<<<<<< HEAD
\label{chap:resultados}
O modelo linear encontrado, considerando a interação entre dois elementos, é disposto a seguir.
=======
O modelo de regressão linear encontrado, considerando a interação entre dois elementos, é disposto a seguir.
>>>>>>> feature/results

\begin{knitrout}
  \definecolor{shadecolor}{rgb}{0.969, 0.969, 0.969}\color{fgcolor}\begin{kframe}
  \begin{verbatim}
  ## 
  ## Call:
  ## lm(formula = tempo ~ (altura + clipe + ad_top + ad_esquerda + 
  ##     ad_direita) + altura * clipe + altura * ad_top + altura * 
  ##     ad_esquerda + altura * ad_direita + clipe * ad_top + clipe * 
  ##     ad_esquerda + clipe * ad_direita + ad_top * ad_esquerda + 
  ##     ad_top * ad_direita + ad_esquerda * ad_direita,
  ##     data = helicoptero)
  ## 
  ## Residuals:
  ##       Min        1Q    Median        3Q       Max 
  ## -0.180625 -0.055312 -0.009375  0.059687  0.120625 
  ## 
  ## Coefficients:
  ##                          Estimate Std. Error t value Pr(>|t|)    
  ## (Intercept)               1.24188    0.07069  17.569 6.99e-12 ***
  ## altura+                   0.42125    0.07903   5.330 6.77e-05 ***
  ## clipe+                   -0.04125    0.07903  -0.522  0.60885    
  ## ad_top+                   0.07875    0.07903   0.996  0.33386    
  ## ad_esquerda+              0.17625    0.07903   2.230  0.04040 *  
  ## ad_direita+               0.19125    0.07903   2.420  0.02779 *  
  ## altura+:clipe+           -0.03250    0.07069  -0.460  0.65186    
  ## altura+:ad_top+           0.09500    0.07069   1.344  0.19771    
  ## altura+:ad_esquerda+     -0.04000    0.07069  -0.566  0.57932    
  ## altura+:ad_direita+      -0.02000    0.07069  -0.283  0.78085    
  ## clipe+:ad_top+           -0.03750    0.07069  -0.531  0.60304    
  ## clipe+:ad_esquerda+       0.06750    0.07069   0.955  0.35382    
  ## clipe+:ad_direita+       -0.14250    0.07069  -2.016  0.06092 .  
  ## ad_top+:ad_esquerda+     -0.13500    0.07069  -1.910  0.07425 .  
  ## ad_top+:ad_direita+      -0.00500    0.07069  -0.071  0.94448    
  ## ad_esquerda+:ad_direita+ -0.21000    0.07069  -2.971  0.00901 ** 
  ## ---
  ## Signif. codes:  0 '***' 0.001 '**' 0.01 '*' 0.05 '.' 0.1 ' ' 1
  ## 
  ## Residual standard error: 0.09996 on 16 degrees of freedom
  ## Multiple R-squared:  0.9161,	Adjusted R-squared:  0.8375 
  ## F-statistic: 11.65 on 15 and 16 DF,  p-value: 6.57e-06
  \end{verbatim}
  \end{kframe}
  \end{knitrout}

Pode-se observar que, para o nosso modelo, as variáveis que possuem relevância estatística, ou seja Pr $<$ 0.05 são: altura (Pr = 6.77e-05), ad\_esquerda (Pr = 0.4040), ad\_direita (Pr = 0.02779) e ad\_esquerda:ad\_direita (Pr = 0.00901). 

Considerando as variáveis que possuem relevância estatística, a equação linear que representa o modelo é descrita da seguinte forma:

\begin{center}
\small  
$
  tempo = média(tempos) + \dfrac{coef(altura)}{2}altura + \dfrac{coef(ad\_esquerda)}{2}ad\_esquerda + \dfrac{coef(ad\_direita)}{2}ad\_direita +  \dfrac{coef(ad\_esquerda:ad\_direita)}{2}ad\_esquerda:ad\_direita
$  
\end{center}
\normalsize
Desta forma, fazendo as devidas substituições, temos que:


  $
     tempo = 1.54 + \dfrac{0.42125}{2}altura + \dfrac{0.17625}{2}ad\_esquerda + \dfrac{0.19125}{2}ad\_direita \\ +  \dfrac{(-0.21)}{2}ad\_esquerda:ad\_direita
  $, logo: 

\begin{equation}
\begin{gathered}  
tempo = 1.54 + 0.210625altura + 0.088125ad\_esquerda + 0.095625ad\_direita - \\
0.105ad\_esquerda:ad\_direita
\label{eq:tempo}
\end{gathered}
\end{equation}



É fácil de verificar na equação \ref{eq:tempo} que as variáveis \textit{altura}, \textit{ad\_direita} e \textit{ad\_esquerda} influenciam positivamente no tempo de queda (possuem coeficientes positivos) enquanto a interação entre as variáveis \textit{ad\_esquerda:ad\_direita} influencia negativamente (possui coeficiente negativo). Desta forma, considerando que nossas variáveis de entrada assumam apenas valores de -1 ou 1, para encontrar o maior valor de tempo de voo devemos atribuir valor positivo às variáveis \textit{altura}, \textit{ad\_direita} e valor negativo à interação \textit{ad\_esquerda:ad\_direita}, o que resulta em:

  $tempo\_max =  1.54 + 0.2106*(1) + 0.0881*(1) + 0.0931*(1) - 0.105*(-1) = 2.04seg$

De forma análoga, para encontrar o menor valor de tempo de voo devemos atribuir valor negativo às variáveis \textit{altura}, \textit{ad\_direita} e \textit{ad\_esquerda} e valor positivo à interação \textit{ad\_esquerda:ad\_direita}, resultando em :

  $tempo\_min =  1.54 + 0.2106*(-1) + 0.0881*(-1) + 0.0931*(-1) - 0.105*(1) = 1.15seg$

Os valores esperados para cada teste de lançamento do helicóptero(\textit{Tempo}), bem como os previstos pelo modelo de regressão linear(\textit{Tempo\_p}) e seus respectivos resíduos podem ser visualizados na Tabela \ref{tab:residuos}.

\begin{table}[H]
  \small
  \center
  \caption{Valores Previstos e Resíduos}
  \begin{tabular}{|c|c|c|}
  \hline
  \textbf{Tempo} & \textbf{Tempo\_p} & \textbf{Resíduo} \\ \hline
  1,27           & 1,241875          & 0,028125         \\ \hline
  1,57           & 1,663125          & -0,093125        \\ \hline
  1,1            & 1,200625          & -0,100625        \\ \hline
  1,7            & 1,589375          & 0,110625         \\ \hline
  1,31           & 1,320625          & -0,010625        \\ \hline
  1,82           & 1,836875          & -0,016875        \\ \hline
  1,3            & 1,241875          & 0,058125         \\ \hline
  1,75           & 1,725625          & 0,024375         \\ \hline
  1,36           & 1,418125          & -0,058125        \\ \hline
  1,92           & 1,799375          & 0,120625         \\ \hline
  1,52           & 1,444375          & 0,075625         \\ \hline
  1,71           & 1,793125          & -0,083125        \\ \hline
  1,4            & 1,361875          & 0,038125         \\ \hline
  1,83           & 1,838125          & -0,008125        \\ \hline
  1,32           & 1,350625          & -0,030625        \\ \hline
  1,74           & 1,794375          & -0,054375        \\ \hline
  1,5            & 1,433125          & 0,066875         \\ \hline
  1,91           & 1,834375          & 0,075625         \\ \hline
  1,23           & 1,249375          & -0,019375        \\ \hline
  1,55           & 1,618125          & -0,068125        \\ \hline
  1,42           & 1,506875          & -0,086875        \\ \hline
  2,04           & 2,003125          & 0,036875         \\ \hline
  1,35           & 1,285625          & 0,064375         \\ \hline
  1,68           & 1,749375          & -0,069375        \\ \hline
  1,44           & 1,399375          & 0,040625         \\ \hline
  1,58           & 1,760625          & -0,180625        \\ \hline
  1,25           & 1,283125          & -0,033125        \\ \hline
  1,73           & 1,611875          & 0,118125         \\ \hline
  1,32           & 1,338125          & -0,018125        \\ \hline
  1,86           & 1,794375          & 0,065625         \\ \hline
  1,17           & 1,184375          & -0,014375        \\ \hline
  1,63           & 1,608125          & 0,021875         \\ \hline
  \end{tabular}
  \label{tab:residuos}
  \end{table}

De posse desses valores, para garantir a veracidade do modelo de regressão linear encontrado deve-se então realizar à análise dos seus resíduos, os quais espera-se possuírem distribuição normal e aleatoriedade em torno da regressão obtida.  A Figura \ref{fig:hist_residuals} exibe o histograma dos resíduos calculados e pode-se observar a distribuição normal dos valores, conforme o esperado.

\begin{figure}[H]
  \caption{Histograma dos resíduos}
  \center
  \includegraphics[scale=0.7]{images/hist_residuals.png}
  \legend {Fonte: Autoria própria.}
  \label{fig:hist_residuals}
\end{figure}

Ao usar a função \textbf{plot()} para o nosso modelo, o primeiro gráfico gerado é o \textit{Residuals vs Fitted}, exibido na Figura \ref{fig:fitted}, que dá uma indicação se há padrões não-lineares nos resíduos. Pode-se verificar na figura que o nosso modelo exibe uma regressão linear atráves de um certo número dos pontos.

\begin{figure}[H]
\caption{\textit{Residuals vs Fitted}}
\center 
\includegraphics[scale=0.48]{images/fitted.png}
\legend {Fonte: Autoria própria.}
\label{fig:fitted}
\end{figure}

Uma outra forma de verificar a distribuição normal dos resíduos é através da Normal Q-Q. Podemos visualizar na Figura \ref{fig:qq} a Normal Q-Q para o nosso modelo, na qual os resíduos seguem próximos à linha reta em diagonal, sendo uma boa indicação de que encontram-se normalmente distríbuidos.

\begin{figure}[H]
  \caption{Normal Q-Q}
  \center
  \includegraphics[scale=0.48]{images/qq.png}
  \legend {Fonte: Autoria própria.}
  \label{fig:qq}
\end{figure}

É necessário verificar também se os resíduos possuem homocedasticidade, ou seja possuam variância comum ao longo da regressão. Pode-se observar no gráfico exibido na Figura \ref{fig:scale} que os resíduos apresentam-se aleatoriamente pela linha, sem concentrar-se nem ao topo nem abaixo da mesma, comportamento que se assemelha ao esperado.

\begin{figure}[H]
  \caption{\textit{Scale-Location}}
  \center 
  \includegraphics[scale=0.48]{images/scale_location.png}
  \legend {Fonte: Autoria própria.}
  \label{fig:scale}
\end{figure}

A partir da análise realizada dos resíduos é possível então inferir que o modelo de regressão linear encontrado na equação \ref{eq:tempo} é válido para o helicóptero de papel utilizado, com as varíaveis \textit{altura}, \textit{ad\_direita} e \textit{ad\_esquerda} possuindo relevância para a resposta observada na variável de saída (\textit{tempo}).
	\chapter{CONCLUSÃO}
\label{chap:conclusao}

Com base na pesquisa elaborada neste relatório, foi possível constatar que o experimento com helicóptero de papel possui eficiência na implementação dos conceitos relativos ao \ac{DOE}. Durante o processo experimental pôde-se observar a importância de uma escolha acertada para as variáveis de entrada e saída, bem como da aleatoriedade de cada teste ao evitar ao máximo, por exemplo, a interferência do vento durante a coleta de dados.

A partir da análise realizada é possível então inferir que o modelo de regressão linear encontrado na equação \ref{eq:tempo} é válido para o helicóptero de papel utilizado, com as variáveis \textit{altura}, \textit{ad\_direita} e \textit{ad\_esquerda} possuindo relevância para a resposta observada na variável de saída (\textit{tempo}), sendo sugerido a otimização destes fatores para a construção de um helicóptero que possua um tempo de voo maior.
	%\include{sections/02referencial}
	%\include{sections/03metodo}
% --------------------------------------------------------------------------
% Referências
	\cleardoublepage
	\titleformat{\chapter}[display]{\vspace*{-24pt}\ABNTEXchapterfont\large\bfseries}{\chaptertitlename\ \thechapter}{12pt}{\Large}
	\bibliography{bibliography}
% --------------------------------------------------------------------------
% Apêndices
	% \apendices
	% \justify
	% %
	% \chapter{Questões de abordagem à pesquisa}
	% \label{apend:quest}
	% %\includepdf[pages={{},-}]{appendix/listquest.pdf}
	% \lipsum[1] % Comentar e adicionar apêndice aqui
	% %
	% \chapter{Um assunto importante}
	% \label{apend:assunto}
	% \lipsum[1] % Comentar e adicionar apêndice aqui
	

% --------------------------------------------------------------------------
% Anexos                                                                     
	% \anexos
	% \justify
	% %
	% \chapter{Outro assunto importante}
	% \label{ann:relant}
	% %\includepdf[pages={{},-}]{annex/manisubanterioridade.pdf}
	% \lipsum[1] % Comentar e adicionar apêndice aqui
	% %
\end{document} 